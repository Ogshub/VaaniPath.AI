\chapter{SYSTEM DESIGN}

\section{Introduction}
System design transforms requirements into a blueprint for implementation. This chapter presents the architectural design, data structures, and diagrams illustrating the My Task Manager application's structure and behavior.

\section{System Architecture}

\subsection{Client-Side Architecture}
The application follows a pure client-side architecture with three logical layers:

\textbf{Presentation Layer (HTML + CSS):} User interface and visual presentation

\textbf{Business Logic Layer (JavaScript):} Core functionality including task management, sorting, and alarm checking

\textbf{Data Layer (localStorage):} Data persistence using browser storage

\section{Data Design}

\subsection{Task Object Structure}
Each task is represented as:
\begin{verbatim}
{
    text: "Task description",
    dateTime: "2026-01-31T14:00" or null,
    completed: false,
    alerted: false
}
\end{verbatim}

\subsection{localStorage Schema}
\textbf{Key:} 'tasks'

\textbf{Value:} JSON array of task objects

\section{Data Flow Diagrams}

\subsection{DFD Level 0}

\begin{figure}[H]
\centering
\includegraphics[width=0.95\textwidth]{DFD0_Original.png}
\caption{Data Flow Diagram - Level 0}
\end{figure}

The context diagram shows the user interacting with the Task Manager system, which stores data in localStorage.

\section{Use Case Diagram}

\begin{figure}[H]
\centering
\includegraphics[width=0.85\textwidth]{UseCase_Original.png}
\caption{Use Case Diagram}
\end{figure}

The use case diagram depicts user interactions: Add Task, View Tasks, Mark Complete, Undo Completion, Delete Task, Set Date/Time, and Receive Alerts.

\section{Class Diagram}

\begin{figure}[H]
\centering
\includegraphics[width=0.95\textwidth]{ClassDiag_Original.png}
\caption{Class Diagram}
\end{figure}

The class diagram shows the main classes and their relationships in the Task Manager system.

\section{User Interface Design}

\subsection{Interface Layout}
\begin{itemize}
\item Header: Application title with emoji icon
\item Input Section: Text input, date picker, time picker, Add button
\item Task List: Vertical list of tasks with action buttons
\end{itemize}

\subsection{Color Scheme}
\begin{itemize}
\item Background: Yellow (\#e6d75e) with star pattern
\item Container: Pink/Magenta (\#b9166d)
\item Task Items: Light gray (\#e9ecef)
\item Buttons: Orange (Add), Green (Complete), Red (Delete)
\end{itemize}

\section{Algorithm Design}

\subsection{Task Sorting Algorithm}
Tasks are sorted by completion status first, then by due date:
\begin{itemize}
\item Incomplete tasks before completed tasks
\item Within each group, earliest due date first
\item Tasks without dates treated as infinite future
\end{itemize}

\textbf{Time Complexity:} O(n log n)

\subsection{Alarm Checking Algorithm}
The system checks for overdue tasks every second:
\begin{itemize}
\item Compare current time with task due time
\item Display alert for overdue incomplete tasks
\item Mark as alerted to prevent duplicates
\end{itemize}

\textbf{Time Complexity:} O(n) per check

\section{Project Timeline}

\begin{figure}[H]
\centering
\includegraphics[width=0.95\textwidth]{Gantt_Original.png}
\caption{Project Gantt Chart}
\end{figure}

The Gantt chart shows the project timeline covering planning, development, testing, and deployment phases.

\section{Security Considerations}
\begin{itemize}
\item Input validation prevents empty tasks
\item Data sanitization prevents XSS attacks
\item All data remains on user's device
\item No external network requests
\end{itemize}

\section{Conclusion}
This chapter presented the comprehensive system design for My Task Manager, including architecture, data structures, and multiple diagram perspectives. This design serves as the blueprint for implementation.
