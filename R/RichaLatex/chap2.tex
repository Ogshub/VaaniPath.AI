\chapter{LITERATURE SURVEY}

\section{Introduction}
This chapter reviews existing task management solutions and explores relevant technologies for the My Task Manager project.

\section{Evolution of Task Management Tools}
Task management evolved from paper-based systems (pre-1990s) to early digital tools (1990s-2000s), then web-based solutions (2000s-2010s), and finally modern cloud-based applications with AI assistance (2010s-present).

\section{Review of Existing Systems}

\subsection{Todoist}
A comprehensive task manager with natural language input, project hierarchies, priority levels, and collaboration features. Strengths include powerful features and cross-platform support. Limitations include required account creation, premium paywall, and complexity for simple use cases.

\subsection{Microsoft To Do}
Integrated with Microsoft 365 ecosystem, offering My Day planning, list sharing, and intelligent suggestions. Free to use but requires Microsoft account and has limited offline functionality.

\subsection{Google Tasks}
Simple task manager integrated with Gmail and Calendar. Extremely easy to use and free with Google account, but has very basic features with no priority levels or recurring tasks.

\subsection{Any.do}
Focused on simplicity with daily planner, voice input, and location-based reminders. Beautiful interface but many features require premium subscription.

\section{Comparative Analysis}

\begin{table}[H]
\centering
\caption{Comparison of Task Management Applications}
\begin{tabular}{|l|c|c|c|c|}
\hline
\textbf{Feature} & \textbf{Todoist} & \textbf{MS To Do} & \textbf{Google Tasks} & \textbf{Any.do} \\
\hline
Account Required & Yes & Yes & Yes & Yes \\
\hline
Offline Mode & Limited & Limited & No & Limited \\
\hline
Privacy & Cloud & Cloud & Cloud & Cloud \\
\hline
Cost & Freemium & Free & Free & Freemium \\
\hline
Complexity & High & Medium & Low & Medium \\
\hline
\end{tabular}
\end{table}

\section{Relevant Technologies}

\subsection{HTML5}
Provides semantic elements, form input types (date, time, text), and local storage API for modern web applications.

\subsection{CSS3}
Offers Flexbox layouts, transitions, animations, and advanced styling for creating modern, responsive interfaces.

\subsection{JavaScript (ES6+)}
Modern JavaScript features including arrow functions, template literals, array methods, and classes power application logic.

\subsection{Web Storage API}
localStorage provides persistent storage with no expiration, typically 5-10 MB capacity, enabling data persistence without backend infrastructure.

\subsection{Document Object Model (DOM)}
Programming interface for HTML documents enabling dynamic content manipulation, event handling, and element creation.

\section{Research in Personal Productivity}

\subsection{Getting Things Done (GTD)}
David Allen's methodology emphasizes capturing all tasks in a trusted external system, validating the need for reliable data persistence and simple task capture.

\subsection{Zeigarnik Effect}
Suggests people remember uncompleted tasks better than completed ones, validating the importance of visual distinction between complete and incomplete tasks.

\section{Identified Gaps}
\begin{itemize}
\item Most solutions store data in cloud, raising privacy concerns
\item Feature bloat overwhelms users needing basic functionality
\item Internet dependency limits usability
\item Account requirements create friction
\item Complex applications require time investment to learn
\end{itemize}

\section{Project Justification}
The My Task Manager project addresses an underserved market segment: users seeking simple, privacy-focused, offline-capable task management. This combination is not well-served by existing solutions, validating the project's unique value proposition.
